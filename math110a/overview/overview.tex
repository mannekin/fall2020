\documentclass{article}

\usepackage{amsmath}
\usepackage{amsfonts}
\usepackage{amsthm}

\newtheorem{theorem}{Theorem}[section]
\newtheorem{corollary}{Corollary}[theorem]
\newtheorem{lemma}[theorem]{Lemma}

\title{Math 110A Overview}
\author{Joseph Wallace}
\date{6 October 2020}

\begin{document}
    
    \maketitle

    \section{Arithmetic in $\mathbb{Z}$ Revisited}

    \subsection{Division Algorithm}
    Division can be conceptualized as how many times a specified sized chunk (the divisor) can be taken out of a number (the dividend).
    The number of times the divisor is taken out of the dividend is known as the quotient.
    The division algorithm is a mathematical way to describe this process.
    \begin{theorem}[Division Algorithm]
        Let a,b be integers with $b>0$. Then there exists unique integers $q$ and $r$ such that
        \begin{equation*}
            a=bq+r \qquad \textrm{and} \qquad 0 \leq r < b
        \end{equation*}
    \end{theorem}
    What this says is that $a$ can be represented as an integer multiple of $b$, with some $r$ left over.
    We have $0 \leq r < b$ because for negative $r$, there are many combinations of $q$ and $r$ that may satisfy the equation.

    \subsection{Divisibility}
    
\end{document}